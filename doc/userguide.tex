\documentclass{article}
\usepackage{graphicx}

\begin{document}

\title{Beam Hardening Correction CarouselFit: User Guide}
\author{Author's Name}

\maketitle

\begin{abstract}
This document is a brief userguide to the Python software package CarousalFit which takes image
data from a number of known samples produced by an X-ray CT machine and fits them to a model of
the beam hardening process which occurs when a broad spectrum source is used image a sample.
This model can then be used to generate corrections appropriate for a single material to convert
the observed attenuation values into the actual attenuation that would be observed for that material
with monochromatic X-rays.
This software is based on the IDL package and ideas described in \cite{davis}.
\end{abstract}

\section{Introduction}
Beam hardening is well known problem that is described in many works, see \cite{davis}.
This software takes as input a number of images of well characterised samples and uses these
to fit a simple model of the expected beam hardening (BH) to the observed data.
The result is an estime of the ``response function'', $R(E)$, which gives the expected output signal
from the detector as a function of X-ray energy for the selected combination of X-ray source, filters
and detector.
Note that due to aging effects of the X-ray source, detector, etc., this function may change over time,
so ideally the calibration measurements should be made before and after each CT scan.
In addition the model allows for variation in the form of $R(E)$ with the number of the scan line.
This can occur due to the way the emitted X-ray spectra is known to depend on the ``take-off'' angle.

The next section describes how to download and run the saoftware.
In section 3 we describe how to set up the necessary file that give
information on the number and type of test images that are used for calibration and the provided
image formats.
Section 4 details how to the run the fitting an post-processing image modules of the software.
Ths first of these fits the model to the data while the second applies the correction either
directly to the CT image data or generates a lookup table to map observed attenuation values to
mono-chromatic attenuation.
Section 5 shows two examples of the use of the software.

\section{Downloading and running the software}

The software is available from the CCPForge repository.
It consists of a Python software pacakge along with a number of data files that are used to help model the X-ray
beams and the material attenuation.
As well as a Python environment the software depends on a number of additional packages being available.
An easy way to access most of the required packages is to download the Anaconda Python environment which is
available for Linux, MacOS and Windows systems from \url{https://www.continuum.io/downloads}.
It is recommended that the user installs this before installing the CarouselFit software.
Alternatively the user may install the required packages in their local Python installation.

The CarouselFit software can be checked out to a suitable directory using the command:
\begin{verbatim}
    svn co https://ccpforge.cse.rl.ac.uk/svn/tomo_bhc/trunk carouselFit
\end{verbatim}
This will create a set of three directories under \verbatim{carouselFit}:
\begin{itemize}
\item \verbatim{src:} this contains the Python source code
\item \verbatim{doc:} this contains documentation of the software
\item \verbatim{test:} this contains several subdirectories with information on attenuation and X-ray spectra.
The source code must be executed from this directory and any updates to the carousel or crown information
should be made in the \verbatim{carouselData} subdirectory.
\end{itemize}

After downloading the software the installation can be checked by running Python in the \verbatim{test} directory
and reading the example script file.
On Linux and MacOS this could be done from a command line. assuming that the Anaconda version of Python is loaded into
the system PATH as:
\begin{verbatim}
  python ../src/runCarouselFit.py
  read script.short
  exit
\end{itemize}
This set of commands should run without producing any error messages, such failure to import some modules.
If a python other than Anaconda has been used it may be necessary to install additional libraries.
Check the documentation for your Python system to see how to do this.

\section{Configuration files}

The original calibation device described in \{cite} was called a carousal as it was built from a set of 9 test samples
arranged between two ciruclar suuports allowing for each of the samples to be imaged individually by the scanner.
The samples would cover the full range of lines in the scanner, but not the full range of each row; typically only
the centre half of each row would be covered by the sample.

A more recent calibration device has been developed at staff at the Research Centre at Harwell (RCaH) which is
known as a crown. This device allows a larger number of samples to be mounted.
In this case the sample usually covers all lines and rows of the image.

The software for analysis of the image data is written in Python and has a simple command line interface.
